\documentclass[a4paper, 12pt, portuguese]{article}

\usepackage[portuguese]{babel}
\usepackage[utf8]{inputenc}
\usepackage{graphicx}



\title{Relatório - Laboratórios de Informática II - 2014/2015}

\author{Luís Fernandes A74748, Mário Silva A75654, Ricardo Certo A75315}

\date{\today}

\begin{document}
\maketitle


\section{Implementação da função R}

\subsection{Função R}
A função \emph{comandoR} recebe um tabuleiro e resolve-o, usando para isso as estratégias E1, E2 e E3.
Recebe como argumentos um tabuleiro e um apontador para a stack e usa também as variáveis i, j para fazer referência às linhas e às colunas
respetivamente, usa a variável flag apenas para saber quando tem de sair fora do while e usa a variável count para contar o
número de casas da matriz onde não aparece nem \emph{'o'} nem \emph{'.'} no tabuleiro passado como argumento.
Enquanto a variável flag é diferente de -1, vai resolvendo o tabuleiro aplicando as estratégias definidas na etapa anterior, sequencialmente.
Com o uso de strchr pretendemos ver em que parte do tabuleiro não existem \emph{'o'} ou \emph{'.'}, e a função devolve-nos um apontador para essa posição,
que caso seja \emph{NULL} (ou seja, estes caracteres não se encontrem nessa posição) incrementa uma unidade à variável count.
Se a nossa variável count for diferente de e.linhas * e.colunas significa que ainda existem os caracteres em cima mencionados no tabuleiro e
por isso a flag permanece igual a 0 e não sai do ciclo while, caso contrário é porque já não existem os caracteres no tabuleiro e por isso o
nosso tabuleiro está resolvido e a função devolve-nos então o tabuleiro resolvido.




\section{Análise do código em Assembly}

\begin{verbatim}
   0x00000020 <+0>:	push   %edi                    = salvaguarda o registo
   0x00000021 <+1>:	push   %esi                    = salvaguarda o registo
   0x00000022 <+2>:	push   %ebx                    = salvaguarda o registo
   0x00000023 <+3>:	mov    0x2720(%esp),%esi       = reserva espaço para os
                                                     argumentos recebidos
                                                     pela função
   0x0000002a <+10>:	cmpb   $0x0,0x2728(%esp)     = compara o que está na
                                                     posição 0x2728(%esp) com 0
   0x00000032 <+18>:	mov    0x2724(%esp),%edx     = atribui um valor a dy
   0x00000039 <+25>:	mov    0x272c(%esp),%eax     = atribui o valor 0 a num
   0x00000040 <+32>:	je     0x90 <contar_segs+112>= se o for igual a 0 salta
                                                     para 0x90 , ou seja vai
                                                     para o else
   0x00000042 <+34>:	lea    -0x1(%eax),%ecx       = faz a operação y=num-1;
                                                     atualiza o valor de y
   0x00000045 <+37>:	xor    %edi,%edi             = inicializa %edi com
                                                     o valor 0
   0x00000047 <+39>:	mov    $0x1,%edx             = atribui o valor 1 a dy
   0x0000004c <+44>:	xor    %eax,%eax             = atribui o valor 0 a num
   0x0000004e <+46>:	test   %esi,%esi             = faz um teste à variável
   0x00000050 <+48>:	jle    0xa0 <contar_segs+128>= se o resultado do teste
                                                     for menor ou igual a 0
                                                     salta para 0xa0
   0x00000052 <+50>:	imul   $0x64,%ecx,%ecx       = multiplica y=y*y*64 ,
                                                     ou seja atribui um novo
                                                     valor a y
   0x00000055 <+53>:	imul   $0x64,%edi,%edi       = multiplica x=*x*64 ,
                                                     ou seja atribui um novo
                                                     valor a x
   0x00000058 <+56>:	add    %eax,%ecx             = atualiza o valor de y ,
                                                     fazendo a operação y+=x
   0x0000005a <+58>:	lea    0x10(%esp),%eax       = atualiza a variável num ,
                                                     com o que está guardado
                                                     em 0x10(%ebp)
   0x0000005e <+62>:	add    %edx,%edi             = atualiza o valor de x ,
                                                     fazendo a operação x+=dy
   0x00000060 <+64>:	add    %eax,%ecx             = atualiza o valor de y ,
                                                     fazendo a operação y+=num
   0x00000062 <+66>:	xor    %edx,%edx             = atualiza o valor de
                                                     dy para 0
   0x00000064 <+68>:	xor    %eax,%eax             = atualiza o valor
                                                     de num para 0
   0x00000066 <+70>:	xchg   %ax,%ax               = troca o valor dos
                                                     operadores
   0x00000068 <+72>:	movzbl (%ecx),%ebx           = move o valor que está na
                                                     pos de memória de
                                                     %ecx (y) para %ebx
   0x0000006b <+75>:	cmp    $0x2e,%bl             = compara o valor $0x2e
                                                     com o valor e bl
   0x0000006e <+78>:	je     0x7c <contar_segs+92> = se for igual a 0 salta
                                                     para 0x7c
   0x00000070 <+80>:	cmp    $0x7e,%bl             = compara o valor $0x7e com
                                                     o valor e bl
   0x00000073 <+83>:	setne  %bl                   = ve se bl é diferente de 0
   0x00000076 <+86>:	cmp    $0x1,%bl              = compara se 1 é igual a
   0x00000079 <+89>:	sbb    $0xffffffff,%eax      =
   0x0000007c <+92>:	add    $0x1,%edx             = adiciona 1 a dy
   0x0000007f <+95>:	add    %edi,%ecx             = soma y+=x , ou seja
                                                     atualiza o valor de y
   0x00000081 <+97>:	cmp    %esi,%edx             = compara o valor de dx com
   0x00000083 <+99>:	jne    0x68 <contar_segs+72> = salta para 0x68 se não
                                                     for 0
   0x00000085 <+101>:	pop    %ebx                  = recupera o registo %ebx
   0x00000086 <+102>:	pop    %esi                  = recupera o registo %esi
   0x00000087 <+103>:	pop    %edi                  = recupera o registo %edi
   0x00000088 <+104>:	ret                          = recupera o SP, o FP e
                                                     o IP e regressa à main
   0x00000089 <+105>:	lea    0x0(%esi,%eiz,1),%esi = resultado de chamar a
                                                     função e_segs
   0x00000090 <+112>:	mov    %edx,%esi             = move o que esta em dy para
   0x00000092 <+114>:	sub    $0x1,%eax             = faz a operação num-=1
   0x00000095 <+117>:	mov    $0x1,%edi             = atualiza o valor de x
                                                     para 1
   0x0000009a <+122>:	xor    %edx,%edx             = atualiza o valor de dy
                                                     para 0
   0x0000009c <+124>:	xor    %ecx,%ecx             = atualiza o valor de y
                                                     para 0
   0x0000009e <+126>:	jmp    0x4e <contar_segs+46> = salta incondicionalmente
                                                     para 0x4e
   0x000000a0 <+128>:	xor    %eax,%eax             = atualiza o valor de num
                                                     para 0
   0x000000a2 <+130>:	jmp    0x85 <contar_segs+101>= salta incondicionalmente
                                                     para 0x85

\end{verbatim}





\end{document}
